% Squelette de rapport de stage
% Compilation recommandée : pdflatex -> biber -> pdflatex -> pdflatex

\documentclass[12pt,a4paper]{report}

% --- Encodage et langue ---
\usepackage[utf8]{inputenc}
\usepackage[T1]{fontenc}
\usepackage[french]{babel}

% --- Mise en page ---
\usepackage{geometry}
\geometry{margin=2.5cm}
\usepackage{microtype}

% --- Figures, tableaux, math ---
\usepackage{graphicx}
\usepackage{caption}
\usepackage{subcaption}
\usepackage{booktabs}
\usepackage{amsmath,amssymb}

% --- Entêtes / pieds de page ---
\usepackage{fancyhdr}
\pagestyle{fancy}
\fancyhf{}
\fancyhead[LE,RO]{\leftmark}
\fancyhead[RE,LO]{\thepage}
\renewcommand{\headrulewidth}{0.4pt}

% --- Liens hypertexte ---
\usepackage[colorlinks=true,linkcolor=black,citecolor=blue,urlcolor=blue]{hyperref}

% --- Commandes utiles ---
\newcommand{\HRule}{\rule{\linewidth}{0.5mm}}

% --- Document ---
\title{Création d'un site web pour geol}
\author{Kelyan TAHARIA\\OPT - NC | UNC\\Encadrant : Adrien SALES}
\date{Janvier 2026}

\begin{document}

% Page de garde personnalisée
\begin{titlepage}
  \centering
  {\scshape\LARGE UNC | OPT - NC \par}
  \vspace{1.5cm}
  {\huge\bfseries Création d'un site web pour geol \par}
  \vspace{1.5cm}
  {\Large Kelyan TAHARIA \par}
  \vfill
  Encadrant : Adrien SALES \\
  OPT - NC | UNC, \today
  \vspace{1cm}
  \HRule
\end{titlepage}

% Résumé
\begin{abstract}
  \noindent Ce rapport présente l'amélioration du site web du projet \texttt{geol} visant à rendre la documentation et les outils d'analyse plus accessibles aux utilisateurs non techniques. Le travail a consisté à clarifier et enrichir les pages de tutoriel, à refondre l'interface visuelle (notamment la page d'accueil), et à implémenter une recherche locale performante et structurée. La méthode adoptée a été itérative : exploration du code existant, modifications ciblées des composants React et des styles, génération d'un index statique pour la recherche, et validations locales via des builds successifs. Les principales réalisations incluent la réécriture de la page "Learn the check command", l'amélioration du moteur de recherche pour dédoublonner et indexer les sections avec ancres, la suppression de pages obsolètes et l'ajout de résumés techniques pour la chaîne d'outils AsciiDoc et l'ingénierie des données. Ces changements améliorent la découverte des fonctionnalités, la navigabilité et la maintenabilité du site. Des pistes d'évolution sont proposées, notamment l'automatisation CI du build et l'optimisation des assets pour la performance.

\bigskip
\begin{center}\textbf{Remerciements}\end{center}
\bigskip
Je remercie chaleureusement Adrien SALES pour son encadrement, ses conseils et sa disponibilité tout au long du stage.
Je remercie également les collègues d'OPT‑NC pour leur aide lors des tests, des revues de code et des validations fonctionnelles. 
Enfin, je remercie l'Office des Postes et Télécommunications de Nouvelle‑Calédonie (OPT‑NC) pour le matériel mis à disposition et pour la confiance témoignée en confiant du matériel informatique à un stagiaire.
\end{abstract}
% Remerciements (déplacés dans l'abstract)

% Table des matières et listes
\cleardoublepage
\tableofcontents
\cleardoublepage
\listoffigures
\listoftables

% --- Corps du rapport ---
\cleardoublepage
\chapter{Introduction}
\section{Contexte général}
Je suis étudiant en deuxième année de la filière CUPGE MP. Du 19 janvier au 8 février 2026, j'effectue un stage à l'Office des Postes et Télécommunications de Nouvelle-Calédonie (OPT-NC). Mon objectif durant ce stage est d'améliorer le site présentant le logiciel Geol, en le rendant plus accessible et attractif pour un large public.

\section{Problématique}
N'ayant pas de connaissances approfondies en informatique, ma problématique est la suivante : est-il possible, même sans compétences techniques, de concevoir et proposer un site web à la fois ludique et ergonomique en s'appuyant sur l'intelligence artificielle pour compenser le manque d'expertise ?

\section{Objectifs}
Les objectifs du stage sont les suivants : partir d'un squelette de site et le transformer en un site très ludique, de sorte qu'une personne sans connaissance en informatique puisse installer et utiliser correctement geol. Présenter geol (un logiciel complexe mais indispensable) de manière claire et accessible pour tous. Optionnel : positionner le site comme site officiel de geol.

\chapter{État de l'art / Contexte technique}
\section{Travaux et références existants}

\chapter{Travaux réalisés}
\section{Méthodologie}
Durée : 7h45 par jour, 5 jours par semaine, pendant 3 semaines — du 19 janvier au 8 février 2026.

Outils utilisés : un ordinateur portable avec Visual Studio Code pour l'édition, GitHub pour le contrôle de version et la diffusion, et un terminal pour les commandes de gestion du site (construction et préproduction).

La méthode adoptée a été itérative : identification des besoins, modification du code source du site, génération d'un index de recherche local, tests locaux et itérations visuelles. Les changements ont été appliqués directement dans le dépôt, construits avec les outils de build du site et vérifiés localement.

\section{Implémentation / Développements}
Travail réalisé (résumé des actions)
\subsection*{Travaux réalisés -- Lundi 19 janvier 2026}
\begin{itemize}
  \item Exploration du dépôt, exécution d'une première build et repérage des pages et composants à modifier.
  \item Planification des changements (visuels, recherche et documentation) et configuration des outils de développement (Git, éditeur, scripts de build).
\end{itemize}

\begin{itemize}
  \item Mise en place d'un volet latéral (menu) qui s'ouvre à gauche et regroupe les entrées importantes : Tutoriel, Blog et Mises à jour (Releases).
  \item Création d'une page "MàJ / Releases" présentant pour chaque version le numéro et un court résumé des changements et améliorations.
  \item Ajout d'une barre de recherche locale : génération d'un index statique des documents et intégration d'une recherche client (Fuse.js) pour des recherches locales, sans dépendre d'un service externe.
  \item Compléments et corrections de la documentation : enrichissement des pages `Products`, introduction au `Tutorial`, compléments pour la chaîne d'outils Pandoc (avec consigne d'installation via Homebrew) et mise à jour de l'ensemble des pages "Tutorial Basics".
  \item Rédaction et clarification de la page "Learn the check command" pour expliquer l'objectif de la commande `geol check` et donner un exemple d'utilisation (par ex. `geol check init`).
  % Remplissage de la page MàJ supprimé : résumé détaillé déplacé ailleurs.
  \item Modification de l'identité visuelle du site : changement du nom affiché dans l'onglet et remplacement du logo dinosaure par le logo officiel `geol` (favicon et logo de la barre de navigation).
\end{itemize}

Remarques techniques et choix : les modifications de l'interface ont été réalisées en éditant les composants React du site, les styles CSS modulaires, et la configuration Docusaurus. La recherche locale a été implémentée en générant un fichier JSON statique avant la construction du site afin d'assurer des performances et une autonomie hors ligne.

Travail accompli en priorité pour rendre le site plus accessible et faciliter la découverte des fonctionnalités de `geol` par des utilisateurs non techniques.

\subsection*{Travaux réalisés -- Mardi 20 janvier 2026}
\begin{itemize}
  \item Réécriture et simplification de la page " Learn the check command " pour la rendre plus compréhensible.
  \item Refonte de l'arrière‑plan de la page d'accueil.
  \item Amélioration de la recherche : indexation des titres avec ancres (\#) et suppression des doublons.
  \item Suppression de la page d'exemple Markdown (`site/src/pages/markdown-page.md`).
  \item Compléments et enrichissement des tutoriels avancés.
\end{itemize}

\subsection*{Travail effectué -- Mercredi 21 janvier 2026}
\begin{itemize}
  \item Les noms ont été ajustés (comme dans le rendu Pandoc) : les noms sont désormais encadrés pour améliorer la lisibilité.
  \item Le \verb|README| du profil GitHub a été rédigé et le profil \verb|dev.to| a été créé.
  \item La documentation de la chaîne d'outils Pandoc et AsciiDoc a été complétée en incorporant les informations fournies par Adrien.
  \item L'image du langage Go a été remplacée par une version en couleur pour renforcer l'identité visuelle.
  \item Les dossiers `.docusaurus` et `build` ont été supprimés du dépôt, commités, puis ajoutés à `.gitignore`.
  \item La barre de recherche a été améliorée : l'indexation a été étendue pour inclure des exemples (par ex. la recherche \verb|list| retrouve maintenant \verb|geol list products| et d'autres exemples pertinents).
  \item La page " Learn the check command " a été clarifiée : le fichier généré par \verb|geol check init| est documenté, les champs produits sont expliqués et le sens des statuts est détaillé.
  \item Le problème d'arrière‑plan de la page d'accueil a été corrigé : le dégradé couvre désormais toute la largeur sans laisser de bande verte.
\end{itemize}

\subsection*{Travaux réalisés -- Jeudi 22 janvier 2026}
\begin{itemize}
  \item Amélioration de la barre de recherche pour de meilleures suggestions et un meilleur dédoublonnage des résultats.
  \item Remplacement des espaces dans les noms de fichiers par des underscores (\verb|_|) afin d'améliorer la compatibilité des chemins et des outils d'automatisation.
  \item Suppression de fichiers inutiles et retrait des répertoires temporaires nommés \verb|tmp| du dépôt.
  \item Élimination des entrées de releases dupliquées sur la page de mises à jour.
  \item Correction de la page d'introduction du tutoriel pour clarifier les étapes d'installation et d'utilisation.
  \item Identification et documentation d'un bug d'installation affectant une version antérieure de \verb|geol| (investigation et pistes de résolution).
  \item Amélioration de la page " Learn the check command " : réorganisation du contenu et clarification des exemples pour la rendre plus intuitive.
\end{itemize}

\subsection*{Travaux réalisés -- Vendredi 23 janvier 2026}
\begin{itemize}
  \item Désactivation de la possibilité de sélectionner ou de copier le texte sur certains éléments de l'interface (barre de navigation, pied de page, sidebars) afin d'améliorer l'expérience visuelle et réduire les usages accidentels.
  \item Mise à jour de la photo de profil sur GitHub.
  \item Ajout de contenus et amélioration du rapport : intégration des notes récentes, corrections typographiques et mise en forme des sections ajoutées.
\end{itemize}

\subsection*{Travaux réalisés -- Lundi 26 janvier 2026}
\begin{itemize}
  \item Optimisation du site : nouveau module client sûr \texttt{lazy-images-client.js} (ne déclenche pas d'erreur SSR). Enregistrement du module dans \texttt{docusaurus.config.js}. Ajout du fichier \texttt{\_headers} pour la mise en cache statique (Netlify). Ajout du marqueur \texttt{\textless{}!-- truncate --\textgreater{}} dans \texttt{blog/2022-12-14.mdx}. Tâche \textit{``Optimiser les performances''} marquée \textit{in-progress}.
  \item Ajout de vidéo(s) dans la section blog.
  \item Ajout de tags et suppression de tags non pertinents.
  \item Ajout d'un bouton pour le choix de la langue (anglais / français).
\end{itemize}

\subsection*{Travaux réalisés -- Mardi 27 janvier 2026}
\begin{itemize}
  \item Traduction complète du site en français et en anglais.
  \item Correction de nombreux bugs (liens non fonctionnels, défauts de traduction, etc.).
  \item Ajustement du bouton de sélection de la langue — correction et reformulation de l'interface si nécessaire.
\end{itemize}

\subsection*{Travaux réalisés -- Mercredi 28 janvier 2026}
\begin{itemize}
  \item Ajout d'un menu interactif permettant d'essayer \texttt{geol} sans l'installer.
  \item Correction et résolution de divers bogues signalés.
\end{itemize}

\subsection*{Travaux réalisés -- Jeudi 29 janvier 2026}
\begin{itemize}
  \item Réorganisation de l'affichage des commandes disponibles.
  \item Terminal (i18n et accessibilité) : localisation des attributs \texttt{aria-label} et du \emph{placeholder} dans \texttt{CLIPlayground.jsx} pour les versions FR/EN.
  \item Images de la page d'accueil : rendu des attributs \texttt{alt} traduisibles dans \texttt{index.js} et ajout des traductions FR dans \texttt{code.json}.
  \item Traductions de l'interface (FR) : corrections de formulations et résolution d'une erreur JSON dans \texttt{code.json}.
  \item Pages traduites (Releases) : suppression des frontmatters dupliqués et nettoyage du fichier \texttt{site-src-pages-releases-md-7ef.md}.
  \item Documentation FR (tutoriels) : relecture et amélioration de la fluidité des fichiers sous le dossier \texttt{current}.
  \item SEO : ajout de meta‑tags de base dans \texttt{docusaurus.config.js}.
  \item Recherche / index : enrichissement du script \texttt{generate-search-index.js} avec les champs \texttt{summary}, \texttt{tags}, \texttt{locale} et \texttt{type}.
\end{itemize}

\subsection*{Travaux réalisés -- Vendredi 30 janvier 2026}
\begin{itemize}
  \item Intervention sur deux sites externes (data centers de Nouville et de Montravel) : ajout de disques durs et remplacement d'une pièce indispensable au fonctionnement d'un relais entre les deux sites.
  \item Correction d'erreurs typographiques, de commentaires et de lignes de code issues d'un ticket fourni par Vin.
\end{itemize}

\chapter{Résultats}
Présenter les résultats obtenus, tableaux et figures.

Les actions menées durant le stage ont permis d'atteindre plusieurs résultats concrets et vérifiables sur le dépôt du site :

\begin{itemize}
  \item Amélioration de la lisibilité et de la navigation des pages de tutoriel (reformulation et réorganisation des sections clefs).
  \item Mise en place d'un index de recherche local plus robuste (ajout de champs \texttt{summary}, \texttt{tags}, \texttt{locale} et \texttt{type}), réduisant les doublons et améliorant la découvrabilité.
  \item Internationalisation et corrections FR : plusieurs pages traduites et reformulées, ainsi que des corrections de traduction UI (fichier \texttt{code.json}).
  \item Clarifications fonctionnelles : modification des pages décrivant le comportement de \texttt{geol} (ex. précision que \texttt{geol} ne stocke pas les métadonnées, et précision sur la commande \texttt{geol export}).
  \item Nettoyage et standardisation : suppression de frontmatters dupliqués et retrait de fichiers temporaires non pertinents.
  \item Exemple de configuration : mise à jour des exemples de configuration (fichier d'exemple `.geol.yaml` dans la documentation tutorielle).
\end{itemize}

\bigskip
\noindent \textbf{Vérification et tests locaux}

Les changements ont été testés localement via des builds successifs du site (commande de build Docusaurus). Les principales vérifications effectuées : rendu des pages modifiées, génération des fichiers `site/static` et `site/build`, contrôle des entrées de l'index de recherche et cohérence des traductions FR/EN.

\bigskip
\noindent \textbf{Tableau récapitulatif (livrables)}

\begin{center}
\begin{tabular}{p{0.45\linewidth} p{0.45\linewidth}}
\toprule
Élément modifié & Description brève \\
\midrule
Pages tutoriels & Réécriture, réorganisation et correction FR/EN \\
Search index & Ajout de summaries/tags/locale/type et nettoyage des doublons \\
Interface & Traductions UI et support des alt traduisibles pour les images \\
Examples & Mise à jour des exemples de configuration et commandes \\
Build & Builds locaux validés (fr + en) \\
\bottomrule
\end{tabular}
\end{center}

\bigskip
Les fichiers générés et modifiés sont disponibles dans le dépôt ; pour une revue détaillée, voir le journal des commits correspondant.

\chapter{Analyse et discussion}
Interprétation des résultats et limites.

L'intervention a permis d'améliorer la qualité perçue de la documentation et de la navigation pour des utilisateurs non techniques. En particulier, la recherche locale est désormais plus complète et renvoie des résultats moins bruités, ce qui facilite la découverte de commandes et d'exemples.

Limites et points à surveiller :
\begin{itemize}
  \item Certaines modifications touchent des fichiers générés (`site/static`, `site/build`) : il est recommandé d'automatiser le processus de build en CI pour éviter des divergences manuelles.
  \item Les captures d'écran et preuves visuelles n'ont pas été incluses ici : il faudra les ajouter dans la version finale du rapport pour illustrer les corrections de bugs et le rendu visuel.
  \item Tests automatisés : l'absence de suite de tests côté documentation rend les régressions possibles lors de futures modifications ; un pipeline de validation (linting, build + vérification d'index) est recommandé.
\end{itemize}

\chapter{Conclusion et perspectives}
Synthèse et pistes pour continuer le travail.

En conclusion, les travaux menés ont rendu le site \texttt{geol} plus accessible et mieux documenté, avec des améliorations notables sur la recherche locale, la cohérence des traductions et la clarté des tutoriels. Pour aller plus loin, je recommande :
\begin{itemize}
  \item Mettre en place une intégration continue (CI) qui exécute le build du site et vérifie l'indexation à chaque push.
  \item Ajouter une petite suite de tests et des checkers (liens morts, validations i18n) afin de détecter automatiquement les régressions.
  \item Compléter le rapport par des captures d'écran et des exemples de sortie (logs) pour illustrer les corrections apportées.
  \item Automatiser la génération de l'index de recherche lors du build pour garantir l'alignement entre sources et fichiers statiques.
\end{itemize}

% --- Annexes ---
\appendix
\chapter{Annexe A}
Éléments complémentaires (ex : code, données, commandes).

% --- Références ---
% Références gérées manuellement. Si vous souhaitez réactiver la
% bibliographie automatique, réajoutez \usepackage{biblatex} et
% \addbibresource{bibliography.bib} puis restaurez l'appel à biber
% dans le `Taskfile.yml`.
\cleardoublepage
\chapter*{Références}
\addcontentsline{toc}{chapter}{Références}

\begin{thebibliography}{9}
\bibitem{geol}
OPT-NC, "geol — Repository for the geol project", GitHub, \url{https://github.com/opt-nc/geol}. Accessed 5 janvier 2026.
\bibitem{adriens_managing_eols}
adriens, "Managing EOLs w/ geol — The impossible 1-mux demo (CNL)", Dev.to, \url{https://dev.to/adriens/managing-eols-w-geol-the-impossible-1-mux-demo-cnl}. Accessed 5 janvier 2026.
\bibitem{docusaurus}
Docusaurus, "Docusaurus — Documentation website generator", \url{https://docusaurus.io/}. Accessed 20 janvier 2026.
\bibitem{homebrew}
Homebrew, "Homebrew — The missing package manager for macOS (and Linux)", \url{https://brew.sh/}. Accessed 20 janvier 2026.
\bibitem{asciidoctor}
Asciidoctor Project, "Asciidoctor — AsciiDoc processor", \url{https://asciidoctor.org/}. Accessed 20 janvier 2026.
\bibitem{endoflife}
EndOfLife Date, "EndOfLife — Current and historical end-of-life dates for software and hardware", \url{https://endoflife.date/}. Accessed 20 janvier 2026.
\end{thebibliography}

\end{document}
